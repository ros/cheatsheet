\documentclass[10pt,landscape]{article}
\usepackage[usenames,dvips,pdftex]{color}
\usepackage{multicol}
\usepackage{calc}
\usepackage{ifthen}
\usepackage[pdftex]{color,graphicx}
\usepackage[landscape]{geometry}
\usepackage{hyperref}
\hypersetup{colorlinks=true, filecolor=black, linkcolor=black, urlcolor=blue, citecolor=black}
\graphicspath{{./images/}}


% To make this come out properly in landscape mode, do one of the following
% 1.
%  pdflatex latexsheet.tex
%
% 2.
%  latex latexsheet.tex
%  dvips -P pdf  -t landscape latexsheet.dvi
%  ps2pdf latexsheet.ps


% If you're reading this, be prepared for confusion.  Making this was
% a learning experience for me, and it shows.  Much of the placement
% was hacked in; if you make it better, let me know...


% 2008-04
% Changed page margin code to use the geometry package. Also added code for
% conditional page margins, depending on paper size. Thanks to Uwe Ziegenhagen
% for the suggestions.

% 2006-08
% Made changes based on suggestions from Gene Cooperman. <gene at ccs.neu.edu>


% To Do:
% \listoffigures \listoftables
% \setcounter{secnumdepth}{0}


% This sets page margins to .5 inch if using letter paper, and to 1cm
% if using A4 paper. (This probably isn't strictly necessary.)
% If using another size paper, use default 1cm margins.
\ifthenelse{\lengthtest { \paperwidth = 11in}}
	{ \geometry{top=.40in,left=.5in,right=.5in,bottom=.40in} }
	{\ifthenelse{ \lengthtest{ \paperwidth = 297mm}}
		{\geometry{top=1cm,left=1cm,right=1cm,bottom=1cm} }
		{\geometry{top=1cm,left=1cm,right=1cm,bottom=1cm} }
	}

% Turn off header and footer
\pagestyle{empty}
 

% Redefine section commands to use less space
\makeatletter
\renewcommand{\section}{\@startsection{section}{1}{0mm}%
                                {-0mm} %plus -.5mm minus -.5mm}%
                                {0.5mm}%x
                                {\normalfont\large\bfseries}}
\renewcommand{\subsection}{\@startsection{subsection}{2}{0mm}%
                                {-0mm}%
                                {0.5ex plus .2ex}%
                                {\normalfont\normalsize\bfseries}}
\renewcommand{\subsubsection}{\@startsection{subsubsection}{3}{0mm}%
                                {-1mm}%
                                {1ex plus .2ex}%
                                {\normalfont\small\bfseries}}
\makeatother

% Define BibTeX command
\def\BibTeX{{\rm B\kern-.05em{\sc i\kern-.025em b}\kern-.08em
    T\kern-.1667em\lower.7ex\hbox{E}\kern-.125emX}}

% Don't print section numbers
\setcounter{secnumdepth}{0}


\setlength{\parindent}{0pt}
\setlength{\parskip}{0pt plus 0.5ex}


% -----------------------------------------------------------------------

\begin{document}

\raggedright
\footnotesize
\begin{multicols}{2}


% multicol parameters
% These lengths are set only within the two main columns
%\setlength{\columnseprule}{0.25pt}
\setlength{\premulticols}{1pt}
\setlength{\postmulticols}{1pt}
\setlength{\multicolsep}{1pt}
\setlength{\columnsep}{2pt}

\begin{center}
     \Large{\textbf{git/hub Cheat Sheet}} \\
\end{center}
\newlength{\MyLen}
\settowidth{\MyLen}{\texttt{letterpaper}/\texttt{a4paper} \ }

%\section{Filesystem Concepts}
%\begin{tabular}{@{}p{\the\MyLen}%
 %               @{}p{\linewidth-\the\MyLen}@{}}
%\texttt{\href{http://www.ros.org/wiki/Packages}{package}}   & The lowest level of ROS software organization. \\
%\texttt{\href{http://www.ros.org/wiki/Manifest}{manifest}}  & Description of a ROS package. \\
%\texttt{\href{http://www.ros.org/wiki/Stack}{stack}} & Collections of ROS packages that form a higher-level library. \\
%\texttt{\href{http://www.ros.org/wiki/Stack Manifest}{stack manifest}}  & Description of a ROS stack.
%\end{tabular}

The purpose of this cheat sheet is to encourage people to start your
project from forking other's repository, not to write
a YET-ANOTHER-ROBOTICS-PACKAGE from scratch.

If you like to know git command itself, please see
\href{https://training.github.com/kit/downloads/github-git-cheat-sheet.pdf}{GIT
  CHEAT SHEET} from \texttt{training.github.com/kit}.

\vspace{2.5mm}
\section{Configure User Information}
\vspace{2.5mm}

1. Create \href{http://github.com}{github} account from
\href{https://github.com/join}{https://github.com/join}.

\vspace{3mm}
2. Install \href{https://github.com/github/hub}{hub} command.
\begin{tabbing}
hu\=b uses \href{http://golang.org/doc/install}{Go development
  environment} to install.\\
\> \texttt{\$ sudo apt-get install golang git-core}\\
\> \texttt{\$ git clone https://github.com/github/hub.git}\\
\> \texttt{\$ cd hub}\\
\> \texttt{\$ ./script/build}\\
\> \texttt{\$ sudo cp hub /usr/local/bin}\\
\end{tabbing}

3. Configure user information for your local environment.

\begin{tabbing}
Se\=ts the name you want to attached to your commit transactions.\\
\> \texttt{\$ git config --global user.name "[name]"}\\  
Sets the email you want to attached to your commit transactions.\\
\> \texttt{\$ git config --global user.email "[email address]"}\\  
Let git to keep a password cashed in memory instead of asking users
every time.\\
\> \texttt{\$ git config --global credential.helper "cache --timeout=3600"}\\  
\end{tabbing}

\section{Start Your Project}
\vspace{2.5mm}

Find any of related project, search
\href{http://www.google.com}{google}, look for
\href{http://wiki.ros.org/}{roswiki}, ask
\href{http://answers.ros.org/questions/}{answers.ros.org}. 
In this cheat sheet, we'll start our project from \href{http://github.com/ros/cheatsheet}{http://github.com/ros/cheatsheet}.

\begin{tabbing}
Cl\=one the project.\\
\> \texttt{\$ hub clone ros/cheatsheet}\\  
Fork the project.\\
\> \texttt{\$ hub fork}\\ 
Type \texttt{hub remote -v} to confirm what was happened.\\
\end{tabbing}

\section{Add New Features, Fix Bugs}
\vspace{2.5mm}

Anytime you write your code, please keep your mind ``When your code is
getting both better and simpler, that is when you know it's
right (From The Cathedral and the Bazaar)''.

\begin{tabbing}
Cr\=eate branch for your work.\\
\> \texttt{\$ git branch -b new\_features}\\  
Write code, check diffs.\\
\> \texttt{\$ git diff}\\ 
If you added new file, add to repository.\\
\> \texttt{\$ git add [file]}\\ 
Check the current status.\\
\> \texttt{\$ git status}\\ 
Revert the file.\\
\> \texttt{\$ git checkout [file]}\\ 
Delete all un-commited files.\\
\> \texttt{\$ git clean -fd}\\ 
Delete everything you worked. CAUTION this remove all of
you change, even if you commited.\\
\> \texttt{\$ git reset --hard origin/master}\\ 
List version history\\
\> \texttt{\$ git log}\\ 
Commit the changes, do not write fix/update/hoge/fuga/ in messages.\\
\> \texttt{\$ git commit -m "descriptable messages"}
\end{tabbing}

\section{Send Pull Request}
\vspace{2.5mm}
It is good idea to create Pull Request and get
feed back, and let CI like jenkins/travis to check your code.
If you're still under working, just write so in the message.
\begin{tabbing}
Pu\=sh branch to your github repository.\\
\> \texttt{\$ git push YOUR\_USER new\_features}\\  
Create Pull Request.\\
\> \texttt{\$ hub pull-request}\\  
If you wan to confirm on browser.\\
\> \texttt{\$ hub browse -- pulls}\\  
\end{tabbing}

\section{If You Get Lost}
\vspace{2.5mm}
\begin{tabbing}
If \=you do not know what you are doing.\\
\> \texttt{\$ git stash}\\
\> \texttt{\$ git checkout master}\\
\> \texttt{\$ git fetch --all}\\
\> \texttt{\$ git reset --hard origin master}\\  
\end{tabbing}

\begin{tabbing}
If \=you w\=ant to delete unnecessary branches\\
\> \texttt{\$ git fetch --all}\\
\> \texttt{\$ git checkout master}\\
\> \texttt{\$ git pull origin master}\\
\> \texttt{\$ git branch -r --merged |  grep YOUR\_USER |  grep -v '>' |  grep -v master |  xargs -L1}\\
\>\> \texttt{|  cut -d"/" -f2- |xargs -n1 git push --delete YOUR\_USER}\\
\> \texttt{\$git branch --merged | grep -v "\textbackslash *" | xargs -n 1 git branch -d}\\
\end{tabbing}

\begin{tabbing}
If \=someone ask you to `please rebase your commit` or `please merge master`\\
\> \texttt{\$ git checkout new\_features}\\  
\> \texttt{\$ git fetch --all}\\
\> \texttt{\$ git rebase origin/master}\\
\> \texttt{\$ git push YOUR\_USER new\_features}\\  
\end{tabbing}

\begin{tabbing}
If \=you want to remove your embarrassing commit history %(Kuro Rekisi), 
,you can modify last N commits by\\
\> \texttt{\$ git rebase -i HEAD\textasciitilde<N>}\\
and, force update to your repository. CAUTION!!! This will remove your
previous commits on github!!!\\
\> \texttt{\$ git push -f YOUR\_USER new\_features}\\  
\end{tabbing}
------------------\\
\scriptsize
This work is licensed under the Creative Commons Attribution 4.0
International License. To view a copy of this license, visit
http://creativecommons.org/licenses/by/4.0/.

Copyright \copyright\ 2014 Kei Okada
%\vspace{3000 mm}

\end{multicols}
\newpage

\begin{center}
paste a ``Re-Inventing the Wheel'' comic from
http://www.willowgarage.com/sites/default/files/blog/201004/willow\_p1\_02s.jpg
\end{center}

\end{document}
